% 导言区域
% article , book ,report ,letter
% 支持中文
\documentclass{ctexart} 
% 注意声明的是graphicx 不是graphics
\usepackage{graphicx}
\usepackage{amsmath}


\title{Latex使用手册以及案例}
\author{Leesure}
\date{\today}



% 正文部分
\begin{document}
% maketitle是将标题显示出来
\maketitle

% figure 浮动体环境
\begin{figure} 
    \centering
    \includegraphics[scale=0.3]{figure/Snipaste_2021-03-03_16-39-12.png}
    \caption{图片的标题}\label{figure_1}
\end{figure}

% 表格的浮动体环境
\begin{table}
    \centering
    \caption{表格的标题}\label{tab_1}
    \begin{tabular}{|l|ccc|r|}
        \hline
        姓名 & 计算机语言 & 数学 & C语言 & 备注\\
        \hline
        李 & 99  & 98 & 87  & 无\\
        \hline
        
    \end{tabular}
    
\end{table}

详情请见图\ref{figure_1}

可以引用表格\ref{tab_1}

hello,\LaTeX

% $符号中的字符是数学公式格式输出,$$双符号表示行间公式


Let  $f(x)$ be defined as $f(x) =x^2+x+1$


    \begin{equation}
        f(x) = x^2+x+1+x^3
        \lim_{x \to \infty}  x^2
        \ddots 
    \end{equation}
    \begin{equation}
        f(x) = x^2+x+1+x^3
        \lim_{x \to \infty}  x^2
        \ddots 
    \end{equation}
    
    % 矩阵的使用,需要引入amsmath宏包 ,\qquad 要用这个
    \[ 
        \begin{matrix}
            0& 1\\
            1& 0 
        \end{matrix} \qquad 
        % pmatrix 环境 小括号
        \begin{pmatrix} 
            0& 1\\
            1& 0 
        \end{pmatrix}  \qquad
        % bmatrix 环境 中括号
        \begin{bmatrix}
            0&1\\
            1&0
        \end{bmatrix} \qquad 
        % 大括号
        \begin{Bmatrix}
            0&1\\
            1&0
        \end{Bmatrix}\qquad
        % 单竖线
        \begin{vmatrix}
            1&0\\
            0&1
        \end{vmatrix}\qquad 
        % 双竖线
        \begin{Vmatrix}
            1&0\\
            0&1 
        \end{Vmatrix}
    \]
    矩阵的使用举例
    % 常用的省略号 
    % \ddots \vdots \dots
    \[
          A= \begin{bmatrix}
              a_{11} & \dots & a_{1n} \\
              \vdots & \ddots & \vdots \\
              0 & \dots & a_{nn}
          \end{bmatrix}_{n \times n}
    \]
    分块矩阵的使用案例:
    \[
          A= \begin{pmatrix}
            %   嵌套举证
            \begin{matrix}
                1&0\\
                0&1
            \end{matrix}
            % & \Large 0 \\ 这种写法没有大写的效果
            & \text{\Large 0} \\
            
            \text{\Large 0} &
            \begin{matrix}
                1&0\\
                -1&0
            \end{matrix}
          \end{pmatrix}
    \]
    三角命令使用案例:
    \[
      \begin{pmatrix}
          a_{11} & a_{12} & \cdots &a_{1n} \\
                 & a_{22} & \cdots & a_{2n} \\
                 &        & \ddots & \vdots \\
        \multicolumn{2}{c}{\raisebox{1.3ex}[0pt]{\Huge 0}} &    &a_{nn}
      \end{pmatrix}  
    \]
    smallmatrix行内小矩阵,但是需要手动写上括号,整个math环境中不能有空行:
      \begin{math}
        \left(
            \begin{smallmatrix}
                x&-y \\
                y&x
            \end{smallmatrix}
        \right)
      \end{math}
      
      
\end{document}