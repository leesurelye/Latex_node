% 导言区域
% article , book ,report ,letter
% 支持中文
\documentclass{ctexart} 
% 注意声明的是graphicx 不是graphics
\usepackage{graphicx}
\usepackage{amsmath}
\usepackage{amssymb}
\usepackage[numbers]{natbib}

% 定义参考文献的格式

\title{Latex使用手册以及案例}
\author{Leesure}
\date{\today}



% 正文部分
\begin{document}
% maketitle是将标题显示出来
\maketitle

% figure 浮动体环境
\begin{figure} 
    \centering
    \includegraphics[scale=0.3]{figure/Snipaste_2021-03-03_16-39-12.png}
    \caption{图片的标题}\label{figure_1}
\end{figure}

% 表格的浮动体环境
\begin{table}
    \centering
    \caption{表格的标题}\label{tab_1}
    \begin{tabular}{|l|ccc|r|}
        \hline
        姓名 & 计算机语言 & 数学 & C语言 & 备注\\
        \hline
        李 & 99  & 98 & 87  & 无\\
        \hline
        
    \end{tabular}
    
\end{table}

详情请见图\ref{figure_1}

可以引用表格\ref{tab_1}

hello,\LaTeX

% $符号中的字符是数学公式格式输出,$$双符号表示行间公式


Let  $f(x)$ be defined as $f(x) =x^2+x+1$


    \begin{equation}
        f(x) = x^2+x+1+x^3
        \lim_{x \to \infty}  x^2
        \ddots 
    \end{equation}
    \begin{equation}
        f(x) = x^2+x+1+x^3
        \lim_{x \to \infty}  x^2
        \ddots 
    \end{equation}
    
    % 矩阵的使用,需要引入amsmath宏包 ,\qquad 要用这个
    \[ 
        \begin{matrix}
            0& 1\\
            1& 0 
        \end{matrix} \qquad 
        % pmatrix 环境 小括号
        \begin{pmatrix} 
            0& 1\\
            1& 0 
        \end{pmatrix}  \qquad
        % bmatrix 环境 中括号
        \begin{bmatrix}
            0&1\\
            1&0
        \end{bmatrix} \qquad 
        % 大括号
        \begin{Bmatrix}
            0&1\\
            1&0
        \end{Bmatrix}\qquad
        % 单竖线
        \begin{vmatrix}
            1&0\\
            0&1
        \end{vmatrix}\qquad 
        % 双竖线
        \begin{Vmatrix}
            1&0\\
            0&1 
        \end{Vmatrix}
    \]
    矩阵的使用举例
    % 常用的省略号 
    % \ddots \vdots \dots
    \[
          A= \begin{bmatrix}
              a_{11} & \dots & a_{1n} \\
              \vdots & \ddots & \vdots \\
              0 & \dots & a_{nn}
          \end{bmatrix}_{n \times n}
    \]
    分块矩阵的使用案例:
    \[
          A= \begin{pmatrix}
            %   嵌套举证
            \begin{matrix}
                1&0\\
                0&1
            \end{matrix}
            % & \Large 0 \\ 这种写法没有大写的效果
            & \text{\Large 0} \\
            
            \text{\Large 0} &
            \begin{matrix}
                1&0\\
                -1&0
            \end{matrix}
          \end{pmatrix}
    \]
    三角命令使用案例:
    \[
      \begin{pmatrix}
          a_{11} & a_{12} & \cdots &a_{1n} \\
                 & a_{22} & \cdots & a_{2n} \\
                 &        & \ddots & \vdots \\
        \multicolumn{2}{c}{\raisebox{1.3ex}[0pt]{\Huge 0}} &    &a_{nn}
      \end{pmatrix}  
    \]
    smallmatrix行内小矩阵,但是需要手动写上括号,整个math环境中不能有空行:
      \begin{math}
        \left(
            \begin{smallmatrix}
                x&-y \\
                y&x
            \end{smallmatrix}
        \right)
      \end{math}
    array环境,类似于表格环境tabular,可以产生横竖线:  

    \section{多行公式的使用}
    使用align环境对公式进行对齐,其中\&符号表示对齐的位置
    \begin{align}
        x &= t+ \cos t +1 \\
        y &= s\sin t
    \end{align}
    使用gather环境来实现公式的多行转换,gather环境将公式编号
    \begin{gather}
        a+b =b+a \\
        a\times b= b\times a
    \end{gather}
    gather*,notag 命令会阻止编号,将公式编号:
    \begin{gather*}
        a+b =b+a \\
        a\times b= b\times a
    \end{gather*}
    不带编号的对齐:
    \begin{align*}
        x &=t   &   x&=\cos t       &   x&=t \\
        y &=2t  &   y&=\sin(t+1)    &   y&= \sin t 
    \end{align*}
    split环境中的对齐采用align环境,且编号在中间
    \begin{equation}
        \begin{split}
            \cos 2x &= \cos^2 x-\sin^2 x \\
                    &= 2\cos^2 x - 1
        \end{split}
    \end{equation}

    cases环境,每行公式使用\& 分割为两部分,通常表示值和后面的条件,注意大写的$\mathbb{Q}$需要
    amssymb宏包的支持
    \begin{equation}
        D(x) = \begin{cases}
            1, & \text{if : } x\in\mathbb{Q}; \\
            0, & \text{if :} x \in \mathbb{R} \setminus \mathbb{Q}
        \end{cases}
    \end{equation}

    \section{参考文献的使用}

    参考文献库需要建立.bib格式的文件,详情请见ref.bib文件。
    在这里将会引用ref.bib中文献库中的文献\\
    LaTeX 参考文献标准选项及其样式共有以下8种:\\
    plain,按字母的顺序排列,比较次序为作者、年度和标题.\\
    unsrt,样式同plain,只是按照引用的先后排序.\\
    alpha,用作者名首字母+年份后两位作标号,以字母顺序排序.\\
    abbrv,类似plain,将月份全拼改为缩写,更显紧凑.\\
    ieeetr,国际电气电子工程师协会期刊样式.\\
    acm,美国计算机学会期刊样式.\\
    siam,美国工业和应用数学学会期刊样式.\\
    apalike,美国心理学学会期刊样式\\
    
    \subsection{使用bibTex管理参考文献库}

    使用thebibliography环境,引用一篇文章使用cite article

    在这里引用文献\cite{article1}中的内容

    在这里引用书籍\cite{book1} 中的内容

    注意使用参考文献库,需要[numbers]{natbib}宏包,并且定义参考文献格式为:plainnat
    引用文献库中的文献\cite{heilman2010good} \\
    不同的文献引用格式 citet 格式:\citet{heilman2010good} \\
    citep格式: \citep{heilman2010good}
    \bibliographystyle{plainnat}
    \bibliography{cited}

    \begin{thebibliography}{99}
        \bibitem{article1}文献作者姓名 \emph{文献名称}[J] 出版社 年份
        \bibitem{book1} 书籍作者姓名 \emph{文献名称} 出版社  年份
    \end{thebibliography}
    
\end{document}